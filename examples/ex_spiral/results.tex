\documentclass{article}
\usepackage[margin=1in]{geometry}

\pagestyle{empty}

%%%%%%%%%%%%%%%%%%%%%%%%%%%%%%%
\usepackage{amsmath, amssymb, amsfonts}
%\usepackage{bbm} % \mathbbm
%%%%%%%%%%%%%%%%%%%%%%%%%%%%%%%
\usepackage{color}
\usepackage{xfrac}
\usepackage{multicol}
\usepackage{afterpage}
\usepackage{ifthen}
%%%%%%%%%%%%%%%%%%%%%%%%%%%%%%%
% figures
%\usepackage{wrapfig}
%\usepackage[usestackEOL]{stackengine}
%\usepackage[export]{adjustbox} % align figures
% \usepackage{subcaption}
% \captionsetup{compatibility=false}
%%%%%%%%%%%%%%%%%%%%%%%%%%%%%%%
% algorithms
%\usepackage{listings}
%\usepackage{algorithm}
%\usepackage{algorithmic}
%%%%%%%%%%%%%%%%%%%%%%%%%%%%%%%
%\usepackage{makecell} % break lines in table cells
%%%%%%%%%%%%%%%%%%%%%%%%%%%%%%%
\usepackage{tikz}
\usepackage{pgfplots}
\usepackage{pgfplotstable}
\pgfplotsset{table/search path={output/data}}
\pgfplotsset{compat=1.15}
%%%%%%%%%%%%%%%%%%%%%%%%%%%%%%%
\usepgfplotslibrary{statistics}
\usetikzlibrary{positioning}
%%%%%%%%%%%%%%%%%%%%%%%%%%%%%%%


% \DeclareMathOperator*{\argmin}{arg\,min}
% \DeclareMathOperator{\Tr}{Tr}
\graphicspath{{./output/images/}}


\begin{document}

\begin{figure}[t]
	\includegraphics[width=0.19\textwidth]{plain-10_result.pdf}
	\includegraphics[width=0.19\textwidth]{1Lip-10_result.pdf}
	\includegraphics[width=0.19\textwidth]{2Lip-5-1Lip-5_result.pdf}
	\includegraphics[width=0.19\textwidth]{10-0_result.pdf}
	\includegraphics[width=0.19\textwidth]{5-5-0.00_result.pdf}
	%%%%%%%
	\\
	\includegraphics[width=0.19\textwidth]{plain-10_result_linear.pdf}
	\includegraphics[width=0.19\textwidth]{1Lip-10_result_linear.pdf}
	\includegraphics[width=0.19\textwidth]{2Lip-5-1Lip-5_result_linear.pdf}
	\includegraphics[width=0.19\textwidth]{10-0_result_linear.pdf}
	\includegraphics[width=0.19\textwidth]{5-5-0.00_result_linear.pdf}
	%%%%%%%
	\\
	\includegraphics[width=0.19\textwidth]{plain-10_spectrum_1.jpg}
	\includegraphics[width=0.19\textwidth]{1Lip-10_spectrum_1.jpg}
	\includegraphics[width=0.19\textwidth]{2Lip-5-1Lip-5_spectrum_1.jpg}
	\includegraphics[width=0.19\textwidth]{10-0_spectrum_1.jpg}
	\includegraphics[width=0.19\textwidth]{5-5-0.00_spectrum_1.jpg}
	\\
	\rule{0.19\textwidth}{0.1pt}
	\rule{0.19\textwidth}{0.1pt}
	\includegraphics[width=0.19\textwidth]{2Lip-5-1Lip-5_spectrum_2.jpg}
	\rule{0.19\textwidth}{0.1pt}
	\includegraphics[width=0.19\textwidth]{5-5-0.00_spectrum_2.jpg}
    \caption{From left to right: plain-10, 1Lip-10, 2Lip-5-1Lip-5, 10-0, 5-5}
\end{figure}


\begin{itemize}
	\item[-] plain-10: 10 layer resnet without spectral normalization
	\item[-] 1Lip-10: 10 layer resnet with spectral normalization of the RHS
	\item[-] 2Lip-5-1Lip-5: 10 layer resnet with with Lipshitz-2 RHS (5 layers) and Lipchitz-1 RHS (5 layers)
	\item[-] 10-0: 10 layer resnet with unstable RHS ($\Re(\lambda) \in [-2,0]$)
	\item[-] 5-5: 10 layer resnet with 5 unstable layers ($\Re(\lambda) \in [-2,0]$) and 5 Lipschitz-1 RHS layers
\end{itemize}

\paragraph{Dataset references}
\begin{enumerate}
	\item Lang, Kevin J., and Michael J. Witbrock. "Learning to tell two spirals apart." Proceedings of the 1988 connectionist models summer school. No. 1989. 1988.
	\item Flake, Gary William. "Square unit augmented radially extended multilayer perceptrons." Neural Networks: Tricks of the Trade. Springer, Berlin, Heidelberg, 1998. 145-163.
\end{enumerate}



\end{document}
